\documentclass{minimal}

\usepackage{parskip}

\begin{document}

Press release for the neural networks/lensing paper

Calvin Sykes\\
\\
\textbf{Neural networks pave the way to a better understanding of Einstein's theories}\\
\\
In the past few years, computational tools known as neural networks have become increasingly used to perform complex tasks, in many cases exceeding the ability of a human.
Neural networks are sophisticated programs that allow a computer to learn from a set of example data, using it to uncover hidden patterns and relationships in new data.\\
\\
Researchers from the Kavli Institute for Particle Astrophysics and Cosmology, and the SLAC National Accelerator Laboratory, have applied these networks to analyse images of strong gravitational lenses.
These are a phenomenon predicted by Einstein's theory of general relativity, whereby light from a distant galaxy is bent by the gravity of a massive object lying along the line of sight.\\
\\
Strong gravitational lenses result in astronomers observing a distorted image, with the distant galaxy's light smeared into an arc or even a complete ring.
Analysis of these images can provide important results for multiple areas of astrophysics: they may be used to trace the elusive dark matter believed to comprise much of the Universe's mass, and comparing them with theoretical predictions allows relativity's accuracy to be tested.\\
\\
In the team's research, submitted as a Letter to the Astrophysical Journal, a model is fitted to the images by finding the values of eight parameters which result in the closest agreement between the model and the data.
They employ a neural network to investigate the amounts by which these parameters may vary while still successfully describing the data, known as the parameters' uncertainties.\\
\\
Accurate determinations of uncertainties are vital for accurately reporting scientific results, meaning the success of this technique may have wide-reaching impact across many fields of research.
A major benefit of the neural network is its ability to outperform traditional methods of determining lensing uncertainties by a factor of more than ten million.
As astronomy enters a new era defined by the enormous volumes of data produced by large-scale surveys of the Universe, this dramatic speed-up is especially welcome.

\end{document}